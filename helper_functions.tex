%\usepackage[a4paper, total={18cm, 25cm}]{geometry}
%, twoside
%making \paragraph behave like \subsubsubsection
\usepackage{titlesec}

\usepackage[hidelinks]{hyperref}

\usepackage{ dsfont }
\usepackage[open]{bookmark} %,openlevel=1

\usepackage{amsthm}

\usepackage{ mathrsfs } 	%for \mathscr

\usepackage{etoolbox}% http://ctan.org/pkg/{amsthm,etoolbox}

\usepackage{booktabs} %for \midrule
\usepackage{amsmath, amssymb, graphics, setspace} %the packages mathematica wants


\usepackage{mathtools}  %to be able to use  \DeclarePairedDelimiter

\usepackage{accents}

\usepackage{esint}

\usepackage{fancyhdr}
\usepackage{lastpage}

\usepackage{amssymb}  %for \mathbb
\usepackage{amsmath}  %for pmatix
\usepackage{scalerel} %for stretchto


%needs packages
\usepackage{esvect}


\usepackage{graphicx}	%for inserting images
\usepackage{float} %for H!
\usepackage{pdfpages}	%for inserting a pdf
%\usepackage{subfig}	%is included in subcaption
\usepackage{caption}	%to fix the size of a caption
\usepackage{subcaption}	%for sub figures
\usepackage{wrapfig}

\usepackage[backend=biber,style=phys,sorting=none]{biblatex}% Ref

\usepackage{subfiles}




\titleformat{\paragraph}
{\normalfont\normalsize\bfseries}{\theparagraph}{1em}{}
\titlespacing*{\paragraph}
{0pt}{3.25ex plus 1ex minus .2ex}{1.5ex plus .2ex}

\titleformat{\subparagraph}
{\normalfont\normalsize\bfseries}{\thesubparagraph}{0.8em}{}
\titlespacing*{\subparagraph}
{0pt}{3.25ex plus 1ex minus .2ex}{1.5ex plus .2ex}

%making sections not have numbers and paragraphs appear as bookmarks
%\makeatletter
%\renewcommand\@seccntformat[1]{}
%\makeatother
\setcounter{secnumdepth}{0}
\setcounter{tocdepth}{5}



\newtheorem{theorem}{Theorem}
\newtheorem{lemma}{Lemma}
\newtheorem{sublemma}{Lemma}[lemma]
\newtheorem{definition}{Definition}
\newtheorem{example}{Example}


\patchcmd{\endproof}% <cmd>
  {\endtrivlist}% <search>
  {\endtrivlist\par\nobreak\vspace*{\dimexpr-\baselineskip-\parskip}\nobreak\noindent\hrulefill}% <replace>
  {}{}% <succes><failure>
  
  \renewcommand{\qedsymbol}{\ensuremath{\blacksquare}}
  
\newcommand{\argmin}{\mathop{\mathrm{arg\,min}}}
\newcommand{\argmax}{\mathop{\mathrm{arg\,max}}}

\newcommand{\arginf}{\mathop{\mathrm{arg\,inf}}}
\newcommand{\argsup}{\mathop{\mathrm{arg\,sup}}}



\DeclarePairedDelimiter\ab{(}{)}  %\ab*{...} will give the \left(...\right)
\DeclarePairedDelimiter\absq{[}{]}  %\abs*{...} will give the \left[...\right]
\DeclarePairedDelimiter\intl{.}{|}  %\intl*{..} for settign integration limits -- only do if not a raw number
\DeclarePairedDelimiter\abrac{\langle}{\rangle} %\abrac*{..} will give <...>

\newenvironment{amatrix}[1]{% defines an augmented matrix used as \begin{amatrix}{n} for n lines before the vertical bar
  \left(\begin{array}{@{}*{#1}{c}|c@{}}
}{%
  \end{array}\right)
} %https://tex.stackexchange.com/questions/2233/whats-the-best-way-make-an-augmented-coefficient-matrix

\newenvironment{amatrixG}[2]{% defines an augmented matrix used as \begin{amatrix}{n}{m} for n lines before the vertical bar and m lines after
  \left(\begin{array}{@{}*{#1}{c}|@{}*{#2}{c}}
}{%
  \end{array}\right)
}

\newcommand{\set}[1]{\left\{ #1 \right \} }

\newcommand{\creg}[1]{\left[ #1 \right] }

\newcommand{\intls}[1]{\Big[ #1 \Big]}

\newcommand{\intll}[1]{#1 \Big|}

\newcommand{\floor}[1]{\left\lfloor #1 \right\rfloor}
\newcommand{\ceil}[1]{\left\lceil #1 \right\rceil}

\newcommand{\abs}[1]{\left| #1 \right|}

\newcommand{\bra}[1]{\left\langle #1 \right|}

\newcommand{\ket}[1]{\left| #1 \right\rangle}

\newcommand{\braket}[2]{\left\langle#1 \middle| #2 \right\rangle}
\newcommand{\braketE}[3]{\left\langle#1 \middle| #2 \middle| #3 \right\rangle}


\newcommand{\partialD}[2]{\dfrac{\partial #1} {\partial #2} }

\newcommand{\partialSD}[2]{\dfrac{\partial^2 #1} {\partial #2 ^2} }

\newcommand{\totalD}[2]{\dfrac{d#1}{d#2}}

\newcommand{\totalSD}[2]{\dfrac{d^2 #1}{d #2^2}}

\newcommand{\divergE}[1]{\dfrac{\partial #1 _x}{\partial x} + \dfrac{\partial #1 _y}{\partial y} + \dfrac{\partial #1 _z}{\partial z}}

\newcommand{\curlM}[1]{\begin{vmatrix}
  \hat i & \hat j & \hat k\\[4pt]
  %
  \dfrac{\partial}{\partial x} & \dfrac{\partial}{\partial y} & \dfrac{\partial}{\partial z}\\[10pt]
  %
  #1 _x & #1 _y & #1 _z\\
\end{vmatrix}}

\newcommand{\curlE}[1]{
  \left( \dfrac{\partial #1_z}{\partial y} - \dfrac{\partial #1 _y}{\partial z}  \right)  \hat i -
  %
  \left( \dfrac{\partial #1 _z}{\partial x} - \dfrac{\partial #1 _x}{\partial z}  \right)  \hat j +
  %
  \left( \dfrac{\partial #1 _y}{\partial x} - \dfrac{\partial #1 _x}{\partial y}  \right)  \hat k
}


\newcommand{\partialSDE}[3] {  %d1/(d2d3)
  \dfrac{\partial^2 #1}{\partial #2 \partial #3}
}

\newcommand{\e}{\textrm e}  %for e = 2.71828..

\newcommand{\dd}{\,\textrm d}

\newcommand{\iiintt}{\iiint\limits}

\newcommand{\iintt}{\iint\limits}

\newcommand{\intt}{\int\limits}

\newcommand{\ointt}{\oint\limits}

\newcommand{\oiintt}{\oiint\limits}

\newcommand{\sign}{\textrm{sign}}

\newcommand{\erf}{\textrm{erf}}

\newcommand{\sinc}{\textrm{sinc}}
\newcommand{\sech}{\textrm{sech}}


\newcommand*{\dt}[1]{%
  \accentset{\mbox{\large\bfseries .}}{#1}}%dot for time derivative
\newcommand*{\ddt}[1]{%
  \accentset{\mbox{\large\bfseries .\hspace{-0.25ex}.}}{#1}}  %double dot for time derivative

\newcommand{\olsi}[1]{\,\overline{\!{#1}}} % overline short italic



\newcommand{\footerFontSize}{10}  %the default latex font size

%Redefine the plain page style
\fancypagestyle{plain} {%
  \fancyhf{}%
  \fancyfoot[L]{\fontsize{\footerFontSize}{\footerFontSize}\selectfont Page \thepage \ of \pageref{LastPage}}%
  \fancyfoot[R]{\fontsize{\footerFontSize}{\footerFontSize}\selectfont Jacob Westerhout}
  \renewcommand{\headrulewidth}{0pt}% Line at header invisible
  \renewcommand{\footrulewidth}{0.4pt}%Line at footer thin
}

\newcommand{\sdfrac}[2]{\mbox{\small$\displaystyle\frac{#1}{#2}$}}
\newcommand{\fdfrac}[2]{\mbox{\footnotesize$\displaystyle\frac{#1}{#2}$}}
\newcommand{\ltfrac}[2]{\mbox{\large$\frac{#1}{#2}$}}

\newcommand\numberthis{\addtocounter{equation}{1}\tag{\theequation}}

%input 1 determines line length
\newcommand{\fullhline}[1]{\par\noindent\makebox[\linewidth]{\rule{\paperwidth}{#1 pt}}\newline}  %line across the whole page
\newcommand{\marginhline}[1]{\par\noindent\rule{\textwidth}{#1 pt}\newline}  %line to the margins



\newcommand{\prob}[1]{\mathbb{P}\left(#1\right)}

\newcommand{\comb}[2]{
\begin{pmatrix}
#1\\#2
\end{pmatrix}
}

\newcommand{\given}{\mathrel{\stretchto{\mid}{3ex}}}

\newcommand{\expec}[1]{\mathbb E \left( #1 \right)}

\newcommand{\unif}[1]{\textrm U\left[#1\right]}

\newcommand{\norm}[1]{\textrm N\left(#1\right)}

\newcommand{\avg}[1]{\left \langle #1 \right \rangle}

\newcommand{\var}[1]{\textrm{Var} \left( #1 \right)}

\newcommand{\cov}[1]{\textrm{Cov} \left( #1 \right)}


\newcommand{\uad}{\hspace{3pt}}




\newcommand{\bvec}[1]{\boldsymbol{#1}}  %vector notated as bold
%{\mathbf{#1}}

\newcommand{\avec}[1]{\vv{#1}} %{\overrightarrow{#1}}

\newcommand{\avecs}[2]{\vv*{#1}{#2}}  %special version of \avec for dealing with subscripts

\newcommand{\inprod}[2]{\left\langle#1,#2 \right\rangle}  %inner product notation

\newcommand{\Tr}[1]{\textrm{Tr}\left(#1\right)}   %trace

\newcommand{\proj}[2]{\textrm{proj}_{#1}\left(#2\right)}

\newcommand{\dotprod}{\cdotp}

\newcommand{\im}{\hat{\imath}}

\newcommand{\jm}{\hat{\jmath}}

\newcommand{\km}{\hat k}

\newcommand{\dive}{\nabla \dotprod}

\newcommand{\matb}[4]{ %2x2 matrix
\begin{pmatrix}
#1 & #2\\
#3 & #4
\end{pmatrix}
}

\newcommand{\spanv}[1]{\textrm{span}\left\{  #1 \right\}}

\newcommand{\transm}[2]{P_{#1 \to #2}}

\newcommand{\tra}{\textrm{tr}} %trace of a matrix


\newcommand{\crossprod}[6]{\begin{vmatrix}
\im & \jm & \km\\
#1 & #2 & #3\\
#4 & #5 & #6
\end{vmatrix}}


\renewcommand{\norm}[1]{\left|\left| #1 \right| \right|}

\newcommand{\atan}{\tan^{-1}}

\newcommand{\vecc}[3]{\begin{pmatrix}
#1\\#2\\#3
\end{pmatrix}}

\newcommand{\vecd}[4]{\begin{pmatrix}
#1\\#2\\#3\\#4
\end{pmatrix}}

%\newcommand{\ker}[1]{\textrm{ker}\left(#1\right)}




\newcommand{\WfigureNC}[2]{\begin{figure}[H]	%1 file loc, 2 width
	\begin{center}
		\includegraphics[width=#2\linewidth]{#1}
	\end{center}
\end{figure}}


\usepackage{cleveref}

\let\cite\parencite 

\pagestyle{plain} %for fancy footer to work correctly